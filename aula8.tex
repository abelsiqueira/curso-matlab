\documentclass{beamer}

\newcommand{\delim}{\line(1,0){290}}
\newcommand{\norma}[1]{\Vert #1 \Vert_2}
\newcommand{\modulo}[1]{\vert #1 \vert}
\newcommand{\prodint}[2]{\langle #1,#2 \rangle}
\newcommand{\barrav}[1]{\line(0,1){#1}}
\newcommand{\ssiz}{\scriptsize}
\newcommand{\rank}{\mbox{rank}}

%\setbeamercolor{cordotitulo}{bg=green!30!black,fg=white}
\useinnertheme{rounded}
\setbeamercolor{structure}{fg=teal!80!blue}
%\useoutertheme{smoothbars}
%\useoutertheme{shadow}
\useoutertheme[height=1 cm,width=1.5 cm]{sidebar}
%\usecolortheme{crane}
%\setbeamerfont{frametitle}{shape=\itshape}
\setbeamercolor{palette primary}{bg=teal!80!blue,fg=white}
\setbeamercolor{palette secondary}{bg=white,fg=white} %cor do logo
\setbeamercolor{palette tertiary}{bg=blue!70!black,fg=white}
\setbeamercolor{palette quaternary}{bg=white,fg=teal!80!blue}
\setbeamercolor{sidebar}{bg=teal!80!blue}
%\setbeamercolor{palette sidebar primary}{bg=red,fg=black}
%\setbeamercolor{palette sidebar secondary}{bg=red,fg=black}
\setbeamercolor{palette sidebar tertiary}{fg=white} %Autor no sidebar
\setbeamercolor{palette sidebar quaternary}{fg=white} %Titulo no sidebar
\setbeamercolor{section in sidebar}{fg=teal!80!blue,bg=white} %A cor naum
\setbeamercolor{subsection in sidebar}{fg=teal!80!blue,bg=white} %A cor naum
%ativa parece ser uma media das duas
%\setbeamercolor{section in sidebar current}{fg=red}
\setbeamercolor{frametitle}{fg=white,bg=teal!80!blue} %mudar
%fg aqui
\setbeamercolor{title}{bg=teal!80!blue,fg=white}
\setbeamerfont{title}{series=\bf}
\setbeamercolor{author}{bg=teal!80!blue,fg=white}
\setbeamerfont{author}{series=\bf}
\setbeamercolor{normal text}{bg=white,fg=teal!50!black}
\logo{\includegraphics[scale=0.06]{matlab_logo.jpg}}

\title{Introdu\c{c}\~ao ao MatLab \\ Aula 8}
\author{Abel Siqueira \\ Kally Chung}
\date{}

\begin{document}

\frame{\titlepage}

\section[T\'opicos especiais]{}

\begin{frame}[fragile]
\frametitle{T\'opicos especiais}

\begin{itemize}
\item<1-> Vamos dar uma r\'apida olhada em alguns t\'opicos especiais
\item<2-> Veremos sistemas lineares, como resolver facilmente e a
decomposi\c{c}\~ao LU.
\item<3-> Tamb\'em iremos ver
\end{itemize}

\end{frame}

\subsection[Sistemas Lineares]{}

\begin{frame}[fragile]
\frametitle{Sistemas Lineares}

Dada uma matriz $A \in \mathbb{R}^{m\times n}$ e um vetor $b \in \mathbb{R}^m$,
queremos encontrar um vetor $x \in \mathbb{R}^n$ que satisfa\c{c}a
$$Ax = b.$$
\pause
\'E preciso saber se esse vetor existe e, caso exista, se \'e \'unico. Vamos
discutir apenas o caso onde a matriz \'e quadrada, isto \'e, quando $m=n$.
\pause

Lembre-se que o posto de uma matriz \'e o n\'umero de colunas linearmente
independentes que ela tem, e denotamos por $\mbox{posto}(A)$ ou
$\mbox{rank}(A)$. A matriz \'e dita ter posto completo se $\rank(A) = n$.
\pause

Para calcular o posto de uma matriz no matlab utilizamos o comando {\tt rank}.
\end{frame}

\begin{frame}[fragile]
\frametitle{Sistemas Lineares}

\begin{itemize}
 \item<1-> Se uma matriz $A$ quadrada tem posto completo ent\~ao qualquer
sistema linear com $A$ tem solu\c{c}\~ao \'unica. O Matlab utiliza o operador
barra invertida (\textbackslash) para encontrar a solu\c{c}\~ao do sistema. Isto
\'e, fazemos \verb+x = A\b+. O operador faz uma s\'erie de verifica\c{c}\~oes
at\'e encontrar alguma maneira de resolver o sistema.

\item<2-> Podemos tamb\'em calcular a inversa da matriz $A$ usando o comando
{\tt inv(A)}. No entanto, calcular a inversa de uma matriz n\~ao \'e eficaz.

\item<3-> Note que \'e muito f\'acil resolver um sistema com a matriz
triangular. Basta resolver uma vari\'avel por vez.
\end{itemize}

\end{frame}

\begin{frame}[fragile]
\frametitle{Sistemas Lineares}

A rotina a seguir resolve um sistema linear com a matriz $L$ triangular
inferior. Consideraremos que a matriz L ter\'a posto completo.
\pause

{\ssiz
\begin{verbatim}
function x = resolve_tri_inf(L,b)

n = size(L,1);
for i = 1:n
  s = 0;
  for j = 1:i-1
    s = s + L(i,j)*x(j)
  end
  x(i) = (b(i) - s)/L(i,i)
end
\end{verbatim}
}

\end{frame}


\begin{frame}[fragile]
\frametitle{Sistemas Lineares}

Tamb\'em podemos utilizar a decomposi\c{c}\~ao LU para resolver o sistema.
Lembre-se que a decomposi\c{c}\~ao LU gera matrizes $L,U$ e $P$ tais que
$$LU = PA,$$
\pause
onde $L$ \'e triangular inferior, $U$ \'e triangular superior, e $P$ \'e uma
matriz de permuta\c{c}\~ao de linhas. Devemos ent\~ao resolver os seguintes
sistemas
$$Ly = Pb, \qquad Ux = y.$$
\pause
Como s\~ao sistemas triangulares, fica f\'acil a resolu\c{c}\~ao.
O Matlab calcula a decomposi\c{c}\~ao LU de uma matriz usando o comando {\tt
[L,U,P] = lu(A)}.
\end{frame}


\begin{frame}[fragile]
\frametitle{Sistemas Lineares}

Ent\~ao, se supomos que j\'a implementamos as rotinas {\tt resolve\_tri\_inf} e
{\tt resolve\_tri\_sup}, que resolvem os sistemas triangular inferior e
triangular superior, respectivamente, ent\~ao podemos resolver o sistema $Ax=b$
fazendo
\pause
\begin{verbatim}
>> [L,U,P] = lu(A);
>> y = resolve_tri_inf(L,P*b);
>> x = resolve_tri_sup(U,y);
\end{verbatim}
\pause

A decomposi\c{c}\~ao LU \'e \'util, por exemplo, quando queremos resolver
v\'arios sistemas com a matriz $A$, mas com vetores $b$ diferentes, em
diferentes instantes.
\pause

O m\'etodo Simplex para programa\c{c}\~ao Linear \'e um exemplo onde isso
acontece.

\end{frame}

\begin{frame}[fragile]
\frametitle{Sistemas Lineares}

\begin{itemize}
\item<1-> O Matlab tamb\'em resolve um sistema linear com v\'arios vetores $b$
ao mesmo tempo, isto \'e, resolve os sistemas
$$Ax^{(1)} = b^{(1)}, Ax^{(2)} = b^{(2)}, \dots, Ax^{(p)} = b^{(p)}.$$
Para isso, basta criar a matriz $B = [b^{(1)} \dots b^{(p)}]$ e escrever
\verb+X = A\B+. Teremos $X = [x^{(1)} \dots x^{(p)}]$.
\item<2-> Os seguintes comandos, j\'a explicados anteriormente, s\~ao bastante
\'uteis, no entanto s\~ao extremamente caros para ser usados: {\tt det, cond,
rank, inv}.
\end{itemize}

\end{frame}

\subsection[An\'alise do programa]{}

\begin{frame}
\frametitle{An\'alise do programa}

\begin{itemize}
\item<1-> Quando acabamos de escrever uma rotina, a coisa mais prov\'avel de acontecer \'e que ela n\~ao funcione.
\item<2-> Vamos ver como o Matlab mostra o erro para n\'os, e os erros mais comuns.
\item<3-> Vamos ver alguns conceitos que podem ser fontes de erro, e que nem sempre s\~ao observados pelo Matlab.
\end{itemize}

\end{frame}

\begin{frame}[fragile]
\begin{itemize}
 \item<1-> Quando voc\^e faz alguma coisa errada ou proibida no Matlab, ele gera uma mensagem de erro. Essa mensagem costuma ser clara quanto ao erro, indicando o problema, a posi\c{c}\~ao onde est\'a o problema, e a fun\c{c}\~ao de onde o problema veio.
\item<2-> Por exemplo, vejamos a seguinte fun\c{c}\~ao que deveria calcular a soma dos elementos do vetor:
\end{itemize}
\pause \pause
{\ssiz
\begin{verbatim}
function S = soma(v)

n = numel(v);
S = 0;
for i = 0:n-1
  S = S + v(i);
end
\end{verbatim}
}

\end{frame}

\begin{frame}[fragile]
\frametitle{An\'alise do programa}

Quando tentamos rodar esse programa, veja o que acontece:
{\scriptsize
\begin{verbatim}
>> v = rand(10,1);
>> S = soma(v)
`Erro! Puta que pariu'
\end{verbatim}
}
\pause

Traduzindo, vemos que o programa est\'a tentando acessar um \'indice do vetor que n\~ao existe. Note que o {\tt for} come\c{c}a com {\tt i = 0}, mas {\tt v(0)} n\~ao existe no Matlab. Esse problema \'e comum entre programadores de outras linguagens. Para resolver, mudamos a linha 5 para \verb+for i = 1:n+, e o programa funcionar\'a normalmente.
\end{frame}

\begin{frame}[fragile]
 \frametitle{An\'alise do programa}

\begin{itemize}
 \item<1-> Quando a fun\c{c}\~ao que cont\'em o erro foi chamado por outra fun\c{c}\~ao, o Matlab indica onde a fun\c{c}\~ao foi chamada na fun\c{c}\~ao ``pai'', e o erro da fun\c{c}\~ao ``filho''.
 \item<2-> Veja, por exemplo, considerando que a fun\c{c}\~ao anterior ainda est\'a errada, a seguinte fun\c{c}\~ao para verificar se dois elementos t\^em o mesmo valor de soma dos elementos:
\end{itemize}
\pause \pause
{\ssiz
\begin{verbatim}
function x = soma_igual(v,w);

if soma(v) == soma(w)
  x = 1;
else
  x = 0;
end
\end{verbatim}
}

\end{frame}

\begin{frame}[fragile]
\frametitle{An\'alise do programa}

Ao rodar esse programa, obtemos o seguinte:
\pause
{\ssiz
\begin{verbatim}
>> v = rand(10,1); w = rand(10,1);
>> soma_igual(v,w)
`Erro! Fudeu de vez'
`Erro! Fudeu de vez'
`Erro! Fudeu de vez'
\end{verbatim}
}
\pause

Ent\~ao conseguimos saber onde est\'a o erro do programa.

\end{frame}

\begin{frame}[fragile]
\frametitle{An\'alise do programa}

\begin{itemize}
 \item<1-> Infelizmente, tamb\'em costumamos errar pois achamos que a fun\c{c}\~ao faz o que queremos, quando na verdade, ela faz algo diferente.
 \item<2-> Nesse caso, geralmente s\'o sabemos que a fun\c{c}\~ao tem algum erro, pois o resultado final n\~ao \'e o que esperamos.
 \item<3-> Tente voltar ao seu programa e tirar alguns ; dos comandos. Assim voc\^e pode ver o que est\'a acontecendo.
 \item<4-> Tamb\'em ajuda incluir alguns {\tt if} para saber se o programa fez o que n\~ao devia.
 \item<5-> Lembre-se tamb\'em dos famosos overflow e underflow, que podem acontecer e n\~ao serem notados.
 \item<6-> O Matlab tamb\'em gera avisos (warning) quando faz algo ilegal, mas que n\~ao precisa travar o progama. Por exemplo, quando tenta dividir por zero, ou inverter uma matriz que n\~ao tem inversa.
\end{itemize}

\end{frame}

\end{document}