\documentclass{beamer}

\newcommand{\ssiz}{\scriptsize}
\newcommand{\delim}{\line(1,0){290}}
\newcommand{\norma}[1]{\Vert #1 \Vert_2}
\newcommand{\modulo}[1]{\vert #1 \vert}
\newcommand{\prodint}[2]{\langle #1,#2 \rangle}
\newcommand{\barrav}[1]{\line(0,1){#1}}

\colorlet{teal}[rgb]{blue!50!green}
%\setbeamercolor{cordotitulo}{bg=green!30!black,fg=white}
\useinnertheme{rounded}
\setbeamercolor{structure}{fg=teal!80!blue}
%\useoutertheme{smoothbars}
%\useoutertheme{shadow}
\useoutertheme[height=1 cm,width=1.5 cm]{sidebar}
%\usecolortheme{crane}
%\setbeamerfont{frametitle}{shape=\itshape}
\setbeamercolor{palette primary}{bg=teal!80!blue,fg=white}
\setbeamercolor{palette secondary}{bg=white,fg=white} %cor do logo
\setbeamercolor{palette tertiary}{bg=blue!70!black,fg=white}
\setbeamercolor{palette quaternary}{bg=white,fg=teal!80!blue}
\setbeamercolor{sidebar}{bg=teal!80!blue}
%\setbeamercolor{palette sidebar primary}{bg=red,fg=black}
%\setbeamercolor{palette sidebar secondary}{bg=red,fg=black}
\setbeamercolor{palette sidebar tertiary}{fg=white} %Autor no sidebar
\setbeamercolor{palette sidebar quaternary}{fg=white} %Titulo no sidebar
\setbeamercolor{section in sidebar}{fg=teal!80!blue,bg=white} %A cor naum
\setbeamercolor{subsection in sidebar}{fg=teal!80!blue,bg=white} %A cor naum
%ativa parece ser uma media das duas
%\setbeamercolor{section in sidebar current}{fg=red}
\setbeamercolor{frametitle}{fg=white,bg=teal!80!blue} %mudar
%fg aqui
\setbeamercolor{title}{bg=teal!80!blue,fg=white}
\setbeamerfont{title}{series=\bf}
\setbeamercolor{author}{bg=teal!80!blue,fg=white}
\setbeamerfont{author}{series=\bf}
\setbeamercolor{normal text}{bg=white,fg=teal!50!black}
\logo{\includegraphics[scale=0.06]{matlab_logo.jpg}}


\title{Introdu\c{c}\~ao ao MatLab \\ Aula 4}
\author{Abel Siqueira \\ Kally Chung}
\date{}

\begin{document}

\frame{\titlepage}

\section[Programa\c{c}\~ao]{}

\begin{frame}
 \frametitle{Programa\c{c}\~ao}

\begin{itemize}
 \item O Matlab tamb\'em pode ser usado como linguagem de programa\c{c}\~ao.
 \item A sintaxe dos comandos \'e bem parecida com C e Fortran.
 \item \'E muito mais f\'acil programar no Matlab pois a maior parte das coisas j\'a est\'a implementada.
 \item N\~ao \'e necess\'ario compilar.
 \item Os comandos apresentados n\~ao precisam ser feitos em arquivos M.
\end{itemize}

\end{frame}

\subsection[Operadores Relacionais]{}

\begin{frame}
\frametitle{Operadores Relacionais}

\begin{itemize}
 \item<1-> O Matlab, assim como outras linguagens de computador, precisa fazer compara\c{c}\~oes entre elementos.
 \item<2-> Ele considera uma proposi\c{c}\~ao verdadeira como 1 e uma proposi\c{c}\~ao falsa como 0.
 \item<3-> Dessa maneira, ao perguntarmos se 1 \'e maior que 0 receberemos a resposta 1, e se perguntarmos se 1 \'e igual a zero, receberemos a resposta 0.
 \item<4-> Os operadores s\~ao:
\end{itemize}
\pause \pause \pause \pause
{\scriptsize
\begin{center}
\begin{tabular}{|c|c|c|c|}
\hline
{\tt > } & Maior & {\tt < } & Menor \\ \hline 
{\tt >= } & Maior ou igual & {\tt <= } & Menor ou igual \\ \hline 
{\tt == } & Igual & {\tt $\sim$= } & Diferente \\ \hline  
\end{tabular}
\end{center}
}
\pause
Note que {\tt =} n\~ao \'e um operador relacional, e sim um operador de atribui\c{c}\~ao.
\end{frame}

\begin{frame}[fragile]
\frametitle{Operadores Relacionais}

{\scriptsize
\begin{center}
\begin{minipage}{3 cm}
\begin{verbatim}
>> 1 > 0
ans =
     1
\end{verbatim}
\end{minipage}
\begin{minipage}{3 cm}
\begin{verbatim}
>> 1 == 0
ans =
     0
\end{verbatim}
\end{minipage}
\begin{minipage}{3 cm}
\begin{verbatim}
>> valor = 2 >= 1
valor =
     1
\end{verbatim}
\end{minipage}
\end{center}
}
\delim
\pause

O Matlab compara vetores e matrizes elemento a elemento:
{\scriptsize
\begin{verbatim}
>> v = [0 1 2 3];
>> w = [2 1 0 -1];
>> v == w
ans =
     0     1     0     0
>> v > w
ans =
     0     0     1     1
>> w >= 0
ans =
     1     1     1     0
\end{verbatim}
}

\end{frame}

\begin{frame}[fragile]
\frametitle{Operadores Relacionais}
Os seguintes comandos s\~ao \'uteis para vetores. Para matrizes, os comandos s\~ao realizados por colunas.
{\scriptsize
\begin{center}
\begin{tabular}{|c|c|}
\hline
{\tt any(<cond\_sobre\_v>)} & Verifica se algum elemento de v satisfaz a \\
& condi\c{c}\~ao. \\ \hline
{\tt all(<cond\_sobre\_v>)} & Verifica se todos os elementos de v satisfazem \\
& a condi\c{c}\~ao. \\ \hline
{\tt find(<cond\_sobre\_v>)} & Retorna um vetor com os \'indices de v que \\
& satisfazem a condi\c{c}\~ao \\ \hline
\end{tabular}
\end{center}}
\pause

Exemplo:
{\scriptsize
\begin{center}
\begin{minipage}{3 cm}
\begin{verbatim}
>> v = [-1 0 1];
>> any(v == 0)
ans =
     1
\end{verbatim}
\end{minipage}
\begin{minipage}{3 cm}
\begin{verbatim}
>> all(v > 0)
ans =
     0
\end{verbatim}
\end{minipage}
\begin{minipage}{3 cm}
\begin{verbatim}
>> find(v <= 0)
ans =
     1     2
\end{verbatim}
\end{minipage}
\end{center}}

\end{frame}

\begin{frame}[fragile]
\frametitle{Operadores L\'ogicos}

\begin{itemize}
\item<1->Os operadores l\'ogicos s\~ao: {\tt \&, |, $\sim$}. S\~ao operadores para senten\c{c}as com resultado 1 ou 0, isto \'e, verdadeiro ou falso.
\item<2->O operador {\tt \&}, chamado de {\bf e}, \'e usado como \verb+A & B+, e \'e verdadeiro apenas se A e B s\~ao verdadeiros.
\item<3->O operador {\tt |}, chamado de {\bf ou}, \'e usado como \verb+A | B+, e \'e verdadeiro se A ou B s\~ao verdadeiros.
\item<4->O operador $\sim$, chamado de {\bf n\~ao}, \'e usado como {\tt $\sim$A}, e \'e verdadeiro se A \'e falso, e falso se A \'e verdadeiro.
\end{itemize}
\pause \pause \pause \pause
{\scriptsize
\begin{verbatim}
>> (1 == 0) | (2 > 1)
ans =
     1
>> (1 == 0) & (2 > 1)
ans =
     0
\end{verbatim}}
\end{frame}

\begin{frame}[fragile]
\frametitle{Operadores L\'ogicos}

O Matlab tamb\'em faz compara\c{c}\~oes l\'ogicos com vetores, elemento a elemento:
{\small
\begin{verbatim}
>> v = 0:6
v = 
     0     1     2     3     4     5     6
>> M = v >= 2
M =
     0     0     1     1     1     1     1
>> m = v <= 5
m =
     1     1     1     1     1     1     0
>> M & m
ans =
     0     0     1     1     1     1     0
\end{verbatim}
}
\end{frame}

\subsection[Os Comandos if, elseif, else]{}

\begin{frame}[fragile]
\frametitle{Os Comandos {\tt if, elseif, else}}

\begin{itemize}
\item<1-> O comando {\tt if} \'e usado quando queremos que algo aconte\c{c}a somente se um determinado requisito \'e satisfeito.
\item<2-> O comando {\tt elseif} \'e usado ap\'os o {\tt if} quando queremos incluir uma nova condi\c{c}\~ao, que acontece se um outro requisito \'e satisfeito e se os requisitos pedidos antes n\~ao foram satisfeitos.
\item<3-> O comando {\tt else} \'e usado ap\'os todos os outros e indica uma a\c{c}\~ao que acontece apenas se nenhum outro requisito foi satisfeito.
\end{itemize}

\end{frame}

\begin{frame}[fragile]
\frametitle{Os Comandos {\tt if, elseif, else}}

A seguir est\'a a sintaxe geral para utilizar o comando:

{\scriptsize
\begin{verbatim}
if <condicao1>
  <acao1>
elseif <condicao2>
  <acao2>
elseif <condicao3>
  <acao3>
.
.
.
elseif <condicaon>
  <acaon>
else
  <acaofinal>
end
\end{verbatim}
}
Pode-se omitir {\tt elseif} e {\tt else}, mas nunca {\tt if} e {\tt end}.

\end{frame}

\begin{frame}[fragile]
\frametitle{Comandos {\tt if, elseif, else}}

Como exemplo, vamos fazer um programa que calcula o sinal de um elemento, isto \'e, a fun\c{c}\~ao deve retornar 1 se o n\'umero for positivo, 0 se o n\'umero for 0, e -1 se o n\'umero for negativo.
\pause
{\scriptsize
\begin{verbatim}
function s = sinal(a)

if a > 0
  s = 1;
elseif a == 0
  s = 0;
else
  s = -1;
end
\end{verbatim}
}

Note que a compara\c{c}\~ao de igualdade \'e feita usando {\tt ==}. Se tentarmos usar {\tt =} o programa gerar\'a erros.
\end{frame}

\begin{frame}[fragile]
\frametitle{Comandos {\tt if, elseif, else}}

Existe ainda o comando {\tt return} que pode ser usado em qualquer parte do programa, mas que tende a ser usado com o {\tt if}. Esse comando para a execu\c{c}\~ao do programa imediatamente. Por exemplo:
\pause

{\small
\begin{verbatim}
function y = calculaAx(A,x)

[m,n] = size(A,2);
if numel(x) ~= n
  return
end
.
.
.

\end{verbatim}
}
\end{frame}

\subsection[O Comando while]{}

\begin{frame}[fragile]
\frametitle{O Comando {\tt while}}

O {\tt while} \'e um comando de repeti\c{c}\~ao (loop). Ele serve para repetir uma a\c{c}\~ao at\'e que uma certa condi\c{c}\~ao seja realizada. A sintaxe usual para o {\tt while} \'e:

\begin{verbatim}
while <condicao>
  <acao>
end
\end{verbatim}

Os comandos descritos em {\tt <acao>} ser\~ao repetidos at\'e que a condi\c{c}\~ao {\tt <condicao>} seja verdadeira.

\end{frame}

\begin{frame}[fragile]
\frametitle{O Comando {\tt while}}

Para exemplificar, vamos calcular qual o menor n\'umero de Fibonacci maior ou igual a um certo valor dado.
\pause

Lembre-se que os n\'umeros de Fibonacci s\~ao dados por $F_1 = F_2 = 1$ e $F_{k+2}=F_{k+1}+F_k$, $k=1,2,\dots$. Suponha $M>1$.
\pause
{\scriptsize
\begin{verbatim}
function k = qualFehmaiorque(M)

F = [1 1 2];
k = 3;
while F(k) < M
  k = k + 1;
  F(k) = F(k-1) + F(k-2);
end
\end{verbatim}}

Enquanto os n\'umeros de Fibonacci estiverem satisfazendo a condi\c{c}\~ao {\tt F(k) < M}, isto \'e, forem menores que M, o programa continuar\'a a gerar n\'umeros de Fibonacci.
\end{frame}

\begin{frame}
\frametitle{O comando {\tt while}}

\begin{itemize}
 \item<1-> Tamb\'em \'e poss\'ivel utilizar outros dois comandos dentro do comando {\tt while}. S\~ao o {\tt break} e o {\tt continue}.
 \item<2-> Estes comandos servem para interromper o loop atual, sem interromper a execu\c{c}\~ao do programa.
 \item<3-> Ao ler o comando {\tt break}, o programa sai imediatamente do loop, mesmo que n\~ao tenha chegado ao fim das instru\c{c}\~oes. No entanto o programa continua logo ap\'os o fim do {\tt end} relacionado ao {\tt while}.
 \item<4-> O comando {\tt continue} funciona como o {\tt break}, mas ele volta para a condi\c{c}\~ao do while.
\end{itemize}

\end{frame}

\subsection[O Comando for]{}

\begin{frame}[fragile]
 \frametitle{O comando {\tt for}}

O comando {\tt for} tamb\'em \'e um comando de repeti\c{c}\~ao. A principal diferen\c{c}a dele para o {\tt while} est\'a no fato de que ele ir\'a fazer um n\'umero de itera\c{c}\~oes j\'a determinado, e vai ter uma vari\'avel com um valor que muda durante cada itera\c{c}\~ao.
\pause

A sintaxe do comando \'e:
\begin{verbatim}
for <variavel>=<vetor>
  <acao>
end
\end{verbatim}
\pause

Os comandos de {\tt <acao>} ser\~ao executados um n\'umero de vezes igual ao tamanho do vetor {\tt <vetor>}. Durante a k-\'esima itera\c{c}\~ao o valor de {\tt <variavel>} ser\'a igual ao k-\'esimo elemento de {\tt <vetor>}.

\end{frame}

\begin{frame}[fragile]
 \frametitle{O Comando {\tt for}}

O modo mais utilizado desse comando segue a cria\c{c}\~ao de vetor descrita na aula 2. As sintaxes mais utilizadas s\~ao
{\scriptsize
\begin{verbatim}
for <variavel> = 1:n
  <acao>
end

for <variavel> = <inicio>:<fim>
  <acao>
end

for <variavel> = <inicio>:<passo>:<fim>
  <acao>
end
\end{verbatim}
}

\end{frame}

\begin{frame}[fragile]
\frametitle{O Comando {\tt for}}
Como exemplo, vamos escrever uma fun\c{c}\~ao que calcula a soma dos elementos de um vetor
\pause
\delim
\begin{verbatim}
function s = soma(v)

n = lenght(v);

s = 0;
for i = 1:n
  s = s + v(i);
end
\end{verbatim}

\pause
\delim

O Comando {\tt for} tamb\'em aceita os comandos {\tt break} e {\tt continue}.

\end{frame}

\begin{frame}[fragile]
\frametitle{O Comando {\tt for}}

Podemos fazer o programa anterior de outra maneira:
\pause

\begin{verbatim}
function s = soma2(v)

s = 0;
for x = v
  s = s + x;
end
\end{verbatim}

\pause

Essa maneira, no entanto, n\~ao \'e a usual.

\end{frame}


\subsection[O Comando switch]{}

\begin{frame}[fragile]
 \frametitle{O Comando {\tt switch}}
O comando {\tt switch} serve para escolher uma a\c{c}\~ao diferente para cada valor de uma vari\'avel. Ele funciona da seguinte forma:
\pause
\delim

{\scriptsize
\begin{minipage}{4 cm}
\begin{verbatim}
switch <variavel>
  case <valor1>
    <acao1>
  case {<valor2>,<valor3>}
    <acao2>
  .
  .
  .
  otherwise
    <acaofinal>
end
\end{verbatim}
\end{minipage}}
\pause
\begin{minipage}{0.2 cm}
\barrav{100}
\end{minipage}
\begin{minipage}{6 cm}
Se {\tt <variavel>} for igual a algum dos valores {\tt <valor1>,<valor2>,\dots}, a a\c{c}\~ao correspondente \'e realizada. Se nenhum valor foi igual ao da vari\'avel, ent\~ao a a\c{c}\~ao {\tt <acaofinal>} \'e realizada.
\end{minipage}

\end{frame}

\begin{frame}[fragile]
\frametitle{O comando {\tt switch}}

Como exemplo considere o seguinte programa que transforma uma medida em cm em outra medida.

{\scriptsize
\begin{verbatim}
function [y,medidanova] = transforma(x,medida)

switch medida
  case 'm'
    y = x/100;
    medidanova = 'm';
  case 'mm'
    y = 10*x;
    medidanova = 'mm';
  case 'pes'
    y = 30,48*x;
    medidanova = 'pes';
  otherwise
  	y = x;
  	medidanova = 'cm';
end
\end{verbatim}
}
\end{frame}

\section[Exerc\'icios]{}

\begin{frame}
 \frametitle{Exerc\'icios}
{\ssiz
\begin{enumerate}
\item Lembre-se que, se $y = Ax$, com $A \in \mathbb{R}^{m\times n}$, ent\~ao
$$y_i = \sum_{j = 1}^na_{ij}x_j, \qquad i = 1,\dots,m.$$
Fa\c{c}a um programa que recebe $A$ e $x$, verifica se as dimens\~oes est\~ao corretas e faz esse c\'alculo.
\item Definimos as normas 1, 2 e infinito para vetores por {\ssiz
$$\Vert x\Vert_1 = \sum_{i = 1}^n \modulo{x_i}, \qquad \norma{x} = \sqrt{\sum_{i=1}^nx_i^2}, \qquad \Vert x\Vert_{\infty} = \max_i \modulo{x_i}$$
}
Fa\c{c}a um programa que recebe um vetor $x$ e uma outra vari\'avel que pode ser {\tt 1, 2} ou {\tt `inf'} e que calcula a norma respectiva em cada caso. N\~ao use {\tt if}.
\end{enumerate}
}

\end{frame}


\begin{frame}
\frametitle{Exerc\'icios}

{\ssiz
\begin{enumerate}
\setcounter{enumi}{2}
\item Para encontrar uma aproxima\c{c}\~ao do n\'umero $\sqrt{c}$, $c > 0$ podemos utilizar a seguinte sequ\^encia:
$$x_{k+1} = \frac{x_k}{2} + \frac{c}{2x_k}, \qquad x_0 = c.$$
Repetimos esse processo at\'e que dois n\'umeros consecutivos dessa sequ\^encia estejam suficientemente pr\'oximos. O \'ultimo elemento obtido \'e a aproxima\c{c}\~ao. Escreva uma rotina que receba $c$ e retorne a aproxima\c{c}\~ao para $\sqrt{c}$. O programa deve verificar se $c > 0$.

\end{enumerate}
}

\end{frame}


\end{document}