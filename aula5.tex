\documentclass{beamer}

\newcommand{\ssiz}{\scriptsize}
\newcommand{\delim}{\line(1,0){290}}
\newcommand{\norma}[1]{\Vert #1 \Vert_2}
\newcommand{\modulo}[1]{\vert #1 \vert}
\newcommand{\prodint}[2]{\langle #1,#2 \rangle}
\newcommand{\barrav}[1]{\line(0,1){#1}}

\usepackage{multicol}

\colorlet{teal}[rgb]{green!50!blue}
%\setbeamercolor{cordotitulo}{bg=green!30!black,fg=white}
\useinnertheme{rounded}
\setbeamercolor{structure}{fg=teal!80!blue}
%\useoutertheme{smoothbars}
%\useoutertheme{shadow}
\useoutertheme[height=1 cm,width=1.5 cm]{sidebar}
%\usecolortheme{crane}
%\setbeamerfont{frametitle}{shape=\itshape}
\setbeamercolor{palette primary}{bg=teal!80!blue,fg=white}
\setbeamercolor{palette secondary}{bg=white,fg=white} %cor do logo
\setbeamercolor{palette tertiary}{bg=blue!70!black,fg=white}
\setbeamercolor{palette quaternary}{bg=white,fg=teal!80!blue}
\setbeamercolor{sidebar}{bg=teal!80!blue}
%\setbeamercolor{palette sidebar primary}{bg=red,fg=black}
%\setbeamercolor{palette sidebar secondary}{bg=red,fg=black}
\setbeamercolor{palette sidebar tertiary}{fg=white} %Autor no sidebar
\setbeamercolor{palette sidebar quaternary}{fg=white} %Titulo no sidebar
\setbeamercolor{section in sidebar}{fg=teal!80!blue,bg=white} %A cor naum
\setbeamercolor{subsection in sidebar}{fg=teal!80!blue,bg=white} %A cor naum
%ativa parece ser uma media das duas
%\setbeamercolor{section in sidebar current}{fg=red}
\setbeamercolor{frametitle}{fg=white,bg=teal!80!blue} %mudar
%fg aqui
\setbeamercolor{title}{bg=teal!80!blue,fg=white}
\setbeamerfont{title}{series=\bf}
\setbeamercolor{author}{bg=teal!80!blue,fg=white}
\setbeamerfont{author}{series=\bf}
\setbeamercolor{normal text}{bg=white,fg=teal!50!black}
\logo{\includegraphics[scale=0.06]{matlab_logo.jpg}}

\title{Introdu\c{c}\~ao ao MatLab \\ Aula 5}
\author{Abel Siqueira \\ Kally Chung}
\date{}

\begin{document}

\frame{\titlepage}

\section[Mais Programa\c{c}\~ao]{}

\begin{frame}
\frametitle{Mais Programa\c{c}\~ao}

\begin{itemize}
\item<1-> Agora que aprendemos a programar, voltemos \`as fun\c{c}\~oes.
\item<2-> Vamos ver primeiramente, uma funcionalidade das fun\c{c}\~oes presente tamb\'em em outras linguagens: Recurs\~ao.
\item<3-> Veremos tamb\'em algumas fun\c{c}\~oes que recebem fun\c{c}\~oes como argumentos, e tamb\'em como faz\^e-las.
\end{itemize}

\end{frame}
\subsection[Recurs\~ao]{}

\begin{frame}
\frametitle{Recurs\~ao}

Recurs\~ao \'e o m\'etodo de se definir uma fun\c{c}\~ao onde se utiliza a pr\'opria fun\c{c}\~ao no meio do c\'odigo. \pause
Um exemplo b\'asico \'e o do c\'alculo do fatorial.
\pause

Lembre-se que, para um n\'umero $n \in \mathbb{N}$, o fatorial $n!$ \'e definido por
$$n! = \left\{
\begin{array}{cc}
1, & \mbox{se } n = 0; \\
n.(n-1).\dots.2.1 & \mbox{para } n > 0;
\end{array}
\right.$$
\pause
Note que podemos escrever o fatorial da seguinte maneira:
$$n! = \left\{
\begin{array}{cc}
1, & \mbox{se } n = 0 \mbox{ ou } n = 1; \\
n.(n-1)! & \mbox{para } n > 1;
\end{array}
\right.$$
\pause
Perceba que o fatorial \'e definido a partir de um fatorial.
\end{frame}

\begin{frame}[fragile]
\frametitle{Recurs\~ao}

Ent\~ao podemos definir a fun\c{c}\~ao para calcular fatorial como
\pause
\begin{verbatim}
function f = fatorial(n)

if (n == 0) || (n == 1)
  f = 1;
elseif n > 1
  f = n*fatorial(n-1);
else
  f = 0;
end
\end{verbatim}
\pause
Este programa calcula o fatorial de um n\'umero natural, ou retorna o valor 0 se o n\'umero n\~ao for natural.

\end{frame}

\subsection[Fun\c{c}\~oes como Argumentos]{}

\begin{frame}[fragile]
\frametitle {Fun\c{c}\~oes como Argumentos}

Algumas fun\c{c}\~oes recebem outras fun\c{c}\~oes como argumentos. Por exemplo, a fun\c{c}\~ao {\tt feval}.
\pause

A sintaxe da fun\c{c}\~ao \'e dada por
\begin{verbatim}
[<var_saida>] = feval(<nome_da_funcao>,<var_entrada>)
\end{verbatim}
\pause

Essa fun\c{c}\~ao simplesmente calcula a fun\c{c}\~ao passada em {\tt <nome\_da\_funcao} com os argumentos  {\tt <var\_entrada>} e retorna os argumentos {\tt <var\_saida>}.

\end{frame}

\begin{frame}[fragile]
\frametitle {Fun\c{c}\~oes como Argumentos}

Por exemplo, podemos usar a fun\c{c}\~ao para calcular a exponencial de um n\'umero:
\pause
\begin{verbatim}
>> f = feval(`exp',5);
f =
     x
\end{verbatim}
\pause

Essa chamada \'e equivalente a
\pause

\begin{verbatim}
>> f = exp(5)
f =
     x
\end{verbatim}

\end{frame}

\begin{frame}
\frametitle {Fun\c{c}\~oes como Argumentos}

O {\tt feval} \'e bastante \'util quando queremos fazer uma fun\c{c}\~ao que recebe uma outra fun\c{c}\~ao gen\'erica. Por exemplo, para calcular a aproxima\c{c}\~ao da integral:
\pause

Lembre-se que podemos aproximar a integral de uma fun\c{c}\~ao por um somat\'orio, com a regra do trap\'ezio:
\pause

{\scriptsize
\begin{eqnarray*}
 \int_{x_1}^{x_n} f(x) dx = \frac{h}{2}[f(x_1)+2f(x_2)+2f(x_3)+\dots+2f(x_{n-1})+f(x_n)],
\end{eqnarray*}
}
onde $h = \frac{x_n-x_1}{n-1}$. Vamos fazer um programa que calcula essa aproxima\c{c}\~ao dados $x_1$,$x_n$ e $n$.

\end{frame}

\begin{frame}[fragile]
 \frametitle {Fun\c{c}\~oes como Argumentos}

Nosso programa deve ter a seguinte cara:
\pause

\begin{verbatim}
function I = integral(funcao,a,b,n)

h = (b-a)/(n-1);
x = a:h:b;
I = 0;
for i = 2:n-1
  I = I + feval(funcao,x(i));
end
I = 2*I;
I = I + feval(funcao,x(1)) + feval(funcao,x(n));
I = I*h/2;
\end{verbatim}

\end{frame}

\begin{frame}
\frametitle {Fun\c{c}\~oes como Argumentos}
Agora vamos programar o m\'etodo de Newton para encontrar zeros de fun\c{c}\~oes
\begin{itemize}
 \item<2-> Inicialmente escreva uma rotina com a fun\c{c}\~ao que voc\^e quer encontrar o zero. Escolha uma fun\c{c}\~ao deriv\'avel e com derivada n\~ao nula perto desse zero.
 \item<3-> Calcule a derivada desta fun\c{c}\~ao.
 \item<4-> Escreva uma rotina com a derivada dessa fun\c{c}\~ao.
 \item<5-> Escreva uma rotina que recebe a fun\c{c}\~ao, a derivada e um ponto inicial e retorna a aproxima\c{c}\~ao do ponto final.
 \item<6-> Lembre que a rotina necessita de um crit\'erio de parada e de um crit\'erio para impedir que fa\c{c}a loop infinito.
\end{itemize}

\end{frame}

\begin{frame}
\frametitle {Fun\c{c}\~oes como Argumentos}
Lembre-se que o m\'etodo de Newton \'e dado por:
\begin{enumerate}
\item<2-> Dados $f = f(x)$, $x_0$, $\varepsilon > 0$ e $k = 0$ fa\c{c}a
\item<3-> Se $\modulo{f(x_k)} < \varepsilon$ PARE
\item<4-> $x_{k+1} = x_k - f(x_k)/f'(x_k)$
\item<5-> $k \leftarrow k+1$ e v\'a para o passo 2
\end{enumerate}

\end{frame}

\begin{frame}[fragile]
\frametitle {Fun\c{c}\~oes como Argumentos}
A sua fun\c{c}\~ao deve ter ficado mais ou menos assim:
\pause

{\ssiz
\begin{verbatim}
function [x,k] = metodo_newton(fun,der,x)
tol = 1e-12;
fx = feval(fun,x);
dx = feval(der,x);
x2 = x - fx/dx;
k = 0;
while abs(fx) >= tol
  x = x2;
  x2 = x2 - fx/dx;
  fx = feval(fun,x);
  dx = feval(der,x);
  k = k + 1;
  if k > 1000
    return
  end
end\end{verbatim}
}

\end{frame}

\begin{frame}[fragile]
\frametitle{Fun\c{c}\~oes como Argumentos}

Veja um exemplo da execu\c{c}\~ao do problema anterior:

{\ssiz
\begin{multicols}{2}
\begin{verbatim}
function f = funcao(x)

f = x^2-1
\end{verbatim}
\begin{verbatim}
function f = derivada(x)

f = 2*x;
\end{verbatim}
\end{multicols}

\pause
\delim

\begin{verbatim}
>> x = 10;
>> [x,k] = metodo_newton(`funcao',`derivada',x)
x =
    1.0000
k =
     8
>> [x,k] = metodo_newton(`sin',`cos',x)
x =
  2.9236e-013
k =
     4
\end{verbatim}
}
\end{frame}

\subsection[Function Handle]{}

\begin{frame}
\frametitle{Function Handle}

\begin{itemize}
 \item<1-> Veremos agora uma ferramenta do Matlab chamada de Function Handle.
 \item<2-> Function Handle serve para criar fun\c{c}\~oes matem\'aticas de uma maneira f\'acil e r\'apida.
 \item<3-> Usando Function Handle n\~ao \'e necess\'ario criar um arquivo m sempre que precisarmos de uma fun\c{c}\~ao matem\'atica.
 \item<4-> Muitas rotinas que aceitam fun\c{c}\~oes por arquivos m tamb\'em aceitam Function Handle, por exemplo o {\tt feval}.
 \item<5-> Assim, podemos criar fun\c{c}\~oes que aceitam tanto arquivos m quanto Function Handle. Por exemplo nossa fun\c{c}\~ao anterior.
\end{itemize}
\end{frame}

\begin{frame}[fragile]
 \frametitle{Function Handle}

Para criar uma fun\c{c}\~ao por function handle usamos a seguinte sintaxe:
\pause

\begin{verbatim}
<nome> = @(<variaveis>) <definicao>
\end{verbatim}
\pause
Exemplos:
{\scriptsize
\begin{verbatim}
>> f = @(x) exp(x);
>> f(1)
ans =
     2.71828
>> g = @(x,n) x^n - 1;
>> g(2,3)
ans =
     8
>> v = @(t,s) [t*s, sqrt(t^2 + s^2)];
>> v(3,4)
ans =
    12     5
\end{verbatim}}

\end{frame}

\begin{frame}[fragile]
\frametitle{Function Handle}

Tamb\'em podemos fazer simplesmente

{\ssiz
\begin{verbatim}
>> f = @exp
>> f(2)
ans =
     7.3891
\end{verbatim}
}

\pause

Tamb\'em \'e poss\'ivel criar Function Handles a partir de outros Funcion Handles

{\ssiz
\begin{verbatim}
>> f = @(x) exp(x)
>> g = @(x) log(f(x))
>> g(2)
ans =
     2
\end{verbatim}
}

\end{frame}

\begin{frame}[fragile]
\frametitle{Function Handle}

Podemos fixar uma vari\'avel e criar um Handle para a outra.
\pause
Veja a nossa rotina do m\'etodo de Newton com Function Handle:
{\ssiz
\begin{verbatim}
>> [x,k] = metodo_newton(@(x) x^2-1, @(x) 2*x, 10)
x =
     1
k =
     8
>> f = @(x,n) x^2-n^2;
>> g = @(x) 2*x
>> [x,k] = metodo_newton(@(x) f(x,1), g, 10)
x =
     1
k =
     8
>> [x,k] = metodo_newton(@(x) f(x,3), @(x) 2*x, 10)
x =
     3
k =
     6
\end{verbatim}
}

\end{frame}


\section[Exerc\'icios]{}

\begin{frame}
\frametitle{Exerc\'icios}

\begin{enumerate}
\item Considere a seguinte seque\^encia:
$$x_{k+1} = \left\{
\begin{array}{ll}
x_k/2, & \mbox{se } x_k \mbox{ \'e par} \\
3x_k + 1, & \mbox{se } x_k \mbox{ \'e \'impar}
\end{array}
\right.$$
Conjectura-se que, para qualquer $x_0 \in \mathbb{N}^{*}$, em algum momento aparece o n\'umero 1 nessa sequ\^encia. Escreva uma rotina que recebe o ponto $x_0$ e que calcula quantas itera\c{c}\~oes demora para encontrar o ponto 1, usando recurs\~ao.
\item Crie uma fun\c{c}\~ao com Function Handle que recebe $a$,$b$,$c$ e $x$ e que retorna $ax^2+bx+c$.
\end{enumerate}

\end{frame}

\begin{frame}
\frametitle{Exerc\'icios}
\begin{enumerate}
\setcounter{enumi}{2}
\item Crie uma fun\c{c}\~ao com Function Handle que calcula uma aproxima\c{c}\~ao para a derivada de uma fun\c{c}\~ao dada. Lembre-se que
$$\lim_{h\rightarrow0}\frac{f(x+h)-f(x)}{h}=f'(x)$$
\item Crie uma rotina que recebe uma fun\c{c}\~ao e sua derivada e um ponto $x$ e calcula um valor $h$ da forma $h = 2^k$ de modo que a aproxima\c{c}\~ao anterior seja precisa at\'e uma toler\^ancia dada. Teste seu algoritmo com:
\begin{enumerate}
\item $f(x) = e^x, \qquad x = 1$
\item $f(x) = 1 - x^2, \qquad x = 4$
\item $f(x) = sin(x), \qquad x = \pi/4$
\item $f(x) = x + 1, \qquad x = 10$
\end{enumerate}
\end{enumerate}
\end{frame}

\end{document}