\documentclass{beamer}

\newcommand{\delim}{\line(1,0){290}}
\newcommand{\ssiz}{\scriptsize}

%\setbeamercolor{cordotitulo}{bg=green!30!black,fg=white}
\useinnertheme{rounded}
\setbeamercolor{structure}{fg=teal!80!blue}
%\useoutertheme{smoothbars}
%\useoutertheme{shadow}
\useoutertheme[height=1 cm,width=1.5 cm]{sidebar}
%\usecolortheme{crane}
%\setbeamerfont{frametitle}{shape=\itshape}
\setbeamercolor{palette primary}{bg=teal!80!blue,fg=white}
\setbeamercolor{palette secondary}{bg=white,fg=white} %cor do logo
\setbeamercolor{palette tertiary}{bg=blue!70!black,fg=white}
\setbeamercolor{palette quaternary}{bg=white,fg=teal!80!blue}
\setbeamercolor{sidebar}{bg=teal!80!blue}
%\setbeamercolor{palette sidebar primary}{bg=red,fg=black}
%\setbeamercolor{palette sidebar secondary}{bg=red,fg=black}
\setbeamercolor{palette sidebar tertiary}{fg=white} %Autor no sidebar
\setbeamercolor{palette sidebar quaternary}{fg=white} %Titulo no sidebar
\setbeamercolor{section in sidebar}{fg=teal!80!blue,bg=white} %A cor naum
\setbeamercolor{subsection in sidebar}{fg=teal!80!blue,bg=white} %A cor naum
%ativa parece ser uma media das duas
%\setbeamercolor{section in sidebar current}{fg=red}
\setbeamercolor{frametitle}{fg=white,bg=teal!80!blue} %mudar
%fg aqui
\setbeamercolor{title}{bg=teal!80!blue,fg=white}
\setbeamerfont{title}{series=\bf}
\setbeamercolor{author}{bg=teal!80!blue,fg=white}
\setbeamerfont{author}{series=\bf}
\setbeamercolor{normal text}{bg=white,fg=teal!50!black}
\logo{\includegraphics[scale=0.06]{matlab_logo.jpg}}

\title{Introdu\c{c}\~ao ao MatLab \\ Aula 2}
\author{Abel Siqueira \\ Kally Chung}
\date{}

\begin{document}

\frame{\titlepage}

\section[Vetores e Matrizes]{}
\begin{frame}[fragile]

  \frametitle{Vetores e Matrizes}

  \begin{itemize}
  \item<1-> Vetores e matrizes s\~ao os principais elementos do Matlab.
  \item<2-> A maior parte dos comandos do Matlab s\~ao para matrizes.
  \item<3-> Tudo s\~ao matrizes, inclusive vetores.
  \item<4-> Utilizamos colchetes para escrever a matriz. Deve-se escrever a matriz por linhas, separando cada elemento da linha por v\'irgula ou espa\c{c}o, e cada mudan\c{c}a de linha por um ponto-e-v\'irgula.
  \item<5-> O Matlab utiliza vetores deitados. Fique atento.
  \end{itemize}

\end{frame}

\begin{frame}[fragile]

\frametitle{Vetores e Matrizes}
Por exemplo, para a matriz
$$\left[\begin{array}{ccc}1 & 2 & 3 \\ 4 & 5 & 6\end{array}\right]$$
devemos escrever {\tt [1 2 3; 4 5 6]} ou {\tt [1,2,3; 4,5,6]}
\pause
\delim
\begin{verbatim}
>> A = [1 2 3; 4 5 6]
A =
     1     2     3
     4     5     6
>> v = [1 0 -1]
v =
     1     0    -1
\end{verbatim}

\end{frame}

\begin{frame}[fragile]
\frametitle{Vetores e Matrizes}
Para acessar o elemento da linha {\tt i} e coluna {\tt j} de uma matriz {\tt A} devemos usar {\tt A(i,j)}. Para vetores, basta apenas um argumento.
\pause
\delim
\begin{center}
\begin{minipage}{4 cm}
\begin{verbatim}
>> A(2,3)
ans =
     6
\end{verbatim}
\end{minipage}
\begin{minipage}{4 cm}
\begin{verbatim}
>> v(3)
ans =
    -1
\end{verbatim}
\end{minipage}
\end{center}
\delim
\pause

Voc\^e tamb\'em pode usar um vetor de \'indices para acessar alguns elementos:
\pause
\delim
\begin{center}
\begin{minipage}{4 cm}
\begin{verbatim}
>> I = [1 3];
>> A(2,I)
ans =
     4     6
\end{verbatim}
\end{minipage}
\begin{minipage}{4 cm}
\begin{verbatim}
>> I = [1 3];
>> v(I)
ans =
     1    -1
\end{verbatim}
\end{minipage}
\end{center}


\end{frame}

\begin{frame}[fragile]
\frametitle{Vetores e Matrizes}
\begin{verbatim}
>> A = [1 2 3; 4 5 6; 7 8 9]
A =
     1     2     3
     4     5     6
     7     8     9
>> I = [2 3]; J = [1 3];
>> A(I,J)
ans =
     4     6
     7     9
\end{verbatim}

\end{frame}
\begin{frame}[fragile]
\frametitle{Vetores e Matrizes}
Voc\^e tamb\'em pode usar {\tt :} para acessar todos os elementos da linha ou coluna:
\pause
\delim
\begin{verbatim}
>> A(2,:)
ans =
     4     5     6
>> A(:,1)
ans =
     1
     4
     7
\end{verbatim}

\end{frame}

\begin{frame}[fragile]
\frametitle{Vetores e Matrizes}
Podamos contatenar duas matrizes facilmente:
\begin{verbatim}
>> A = [1 2; 3 4];
>> B = [5 6; 7 8];
>> C = [A B]
C =
     1     2     5     6
     3     4     7     8
>> D = [A;B]
D =
     1     2
     3     4
     5     6
     7     8
\end{verbatim}
\end{frame}

\subsection[Opera\c{c}\~oes Matriciais]{}

\begin{frame}[fragile]
\frametitle{Opera\c{c}\~oes Matriciais}

As opera\c{c}\~oes matriciais est\~ao todas implementadas no Matlab, mas fique atento \`as dimens\~oes:

\delim
{\scriptsize
\begin{center}
\begin{minipage}{4 cm}
\begin{verbatim}
>> A = [1 2 3; 4 5 6]
A =
     1     2     3
     4     5     6
>> B = [1 2; 3 4; 5 6]
B =
     1     2
     3     4
     5     6
\end{verbatim}
\end{minipage}\hspace{1.5 cm}
\begin{minipage}{4 cm}
\begin{verbatim}
>> A*B
ans =
    22    28
    49    64
>> B*A
ans =
     9    12    15
    19    26    33
    29    40    51
\end{verbatim}
\end{minipage}
\end{center}
\delim
\begin{center}
\begin{minipage}{6 cm}
\begin{verbatim}
 >> A*A
??? Error using ==> mtimes
Inner matrix dimensions must agree.
\end{verbatim}
\end{minipage}
\end{center}
}

\end{frame}

\begin{frame}
\frametitle{Opera\c{c}\~oes Matriciais}

Al\'em das opera\c{c}\~oes matriciais b\'asicas {\tt +, -, *, \textasciicircum} (Note que n\~ao existe divis\~ao entre matrizes) o Matlab tamb\'em suporta as chamadas opera\c{c}\~oes Strassen. Essas s\~ao opera\c{c}\~oes entre matrizes de mesma dimens\~ao relacionando cada elemento de uma matriz com o elemento na posi\c{c}\~ao correspondente da outra. As opera\c{c}\~oes Strassen s\~ao: {\tt .*, ./, .\textbackslash, .\textasciicircum}.
\pause
Exemplo:
\begin{eqnarray*}
\left[
\begin{array}{ccc}
1 & 2 & 3 \\
4 & 5 & 6
\end{array}
\right].*
\left[
\begin{array}{ccc}
3 & 0 & -1 \\
2 & 3 & 5
\end{array}
\right] & = &
\left[
\begin{array}{ccc}
1*3 & 2*0 & 3*(-1) \\
4*2 & 5*3 & 6*5
\end{array}
\right] \\
& = &
\left[
\begin{array}{ccc}
3 & 0 & -3 \\
8 & 15 & 30
\end{array}
\right]
\end{eqnarray*}
\pause
Para calcular a transposta de uma matriz, utilizamos {\tt '}, isto \'e, a transposta de {\tt A} \'e {\tt A'}.
\end{frame}

\begin{frame}[fragile]
\frametitle{Opera\c{c}\~oes Matriciais}
\begin{verbatim}
>> A = [1 2 3; 4 5 6];
>> B = [3 0 -1; 2 3 5];
>> C = A.*B
C =
     3     0    -3
     8    15    30
>> D = C'
D =
     3     8
     0    15
    -3    30
\end{verbatim}
\end{frame}

\subsection[Cria\c{c}\~ao de Vetores]{}

\begin{frame}[fragile]
\frametitle{Cria\c{c}\~ao de Vetores}
\begin{itemize}
\item<1-> Podemos sempre utilizar o m\'etodo anterior para criar vetores e matrizes, mas se o vetor ou matriz \'e muito grande, o trabalho para escrev\^e-lo \'e exaustivo. Em casos particulares podemos utilizar recursos do Matlab para essa cria\c{c}\~ao.
\item<2-> O caso mais comum de vetores \'e quando os n\'umeros est\~ao dispostos em uma sequ\^encia de progress\~ao aritm\'etica, isto \'e, quando os intervalos s\~ao igualmente espa\c{c}ados.
\item<3-> Podemos usar {\tt :} para criar facilmente estes vetores, fazendo {\tt <inicio>:<fim>} ou {\tt <inicio>:<passo>:<fim>}. No primeiro caso {\tt <passo>} = 1.
\item<4-> O vetor gerado come\c{c}a com {\tt <inicio>} e incrementa {\tt <passo>} a cada posi\c{c}\~ao at\'e um n\'umero n\~ao ultrapasse {\tt <fim>}
\end{itemize}

\end{frame}

\begin{frame}[fragile]
\frametitle{Cria\c{c}\~ao de Vetores}
{\ssiz
\begin{verbatim}
>> 1:6
ans =
     1     2     3     4     5     6
\end{verbatim}
\pause
\begin{verbatim}
>> 1:6.5
ans =
     1     2     3     4     5     6
\end{verbatim}
\pause
\begin{verbatim}
>> 4:-0.1:3.5
ans =
    4.0000    3.9000    3.8000    3.7000    3.6000    3.5000
\end{verbatim}
}
\end{frame}

\subsection[Fun\c{c}\~oes para Vetores]{}
\begin{frame}
\frametitle{Fun\c{c}\~oes para Vetores}

As seguintes fun\c{c}\~oes para vetores s\~ao extremamente \'uteis:
{\scriptsize
\begin{center}
\begin{tabular}{|c|c|}
\hline
Comando & Explica\c{c}\~ao \\ \hline
{\tt linspace(a,b,N)} & Gera N n\'umeros, O primeiro a, o \'ultimo b, \\
& e igualmente espa\c{c}ados. \\ \hline
{\tt logspace(a,b,N)} & Gera N n\'umeros, O primeiro $10^a$, o \'ultimo $10^b$, e\\
& espa\c{c}ados logar\'itmicamente.\\ \hline
{\tt lenght(v)} & Calcula o tamanho de v. \\ \hline
{\tt dot(v,w)} & Calcula o produto interno de v e w. \\ \hline
{\tt cross(v,w)} & Calcula o produto externo de v e w. \\ \hline
{\tt norm(v,a)} & Calcula uma norma de v. a pode ser 1, 2 ou `inf'. \\
& Pode-se omitir a, obtendo assim a norma 2. \\ \hline
{\tt sum(v)} & Retorna a soma todos os elementos de v. \\ \hline
{\tt prod(v)} & Retorna a multipli\c{c}\~ao todos os elementos de v. \\ \hline
{\tt [M,k] = max(v)} & M \'e o valor m\'aximo de v e k \'e sua posi\c{c}\~ao. \\ \hline
{\tt [M,k] = min(v)} & M \'e o valor m\'inimo de v e k \'e sua posi\c{c}\~ao. \\ \hline
\end{tabular}
\end{center}
}
\end{frame}

\subsection[Cria\c{c}\~ao de Matrizes]{}

\begin{frame}
\frametitle{Cria\c{c}\~ao de Matrizes}

\begin{itemize}
\item<1-> Dentre os comandos poss\'iveis para se gerar matrizes, os tr\^es mais famosos s\~ao {\tt ones, zeros} e {\tt eye}.
\item<2-> O comando {\tt ones(m,n)} gera uma matriz m por n onde todos os elementos s\~ao iguais a 1. Pode-se omitir o n, gerando assim uma matriz m por m.
\item<3-> O comando {\tt zeros(m,n)} gera uma matriz m por n onde todos os elementos s\~ao iguais a 0. Pode-se omitir o n, gerando assim uma matriz m por m.
\item<4-> O comando {\tt eye(m,n)} gera uma matriz com uma submatriz quadrada sendo a identidade. O tamanho da submatriz \'e $\min\{m,n\}$
\end{itemize}

\end{frame}

\subsection[Fun\c{c}\~oes para Matrizes]{}
\begin{frame}
\frametitle{Fun\c{c}\~oes para Matrizes}

As seguintes fun\c{c}\~oes para matrizes s\~ao bem \'uteis:
{\scriptsize
\begin{center}
\begin{tabular}{|c|c|}
\hline
Comandos & Explica\c{c}\~ao \\ \hline
{\tt det(A)} & Calcula o determinante da matriz A \\ \hline
{\tt rank(A)} & Calcula o posto da matriz A \\ \hline
{\tt inv(A)} & Calcula a inversa da matriz A \\ \hline
{\tt rand(m,n)} & Gera uma matriz m por n onde os \\
& elementos s\~ao n\'umeros aleat\'orios entre 0 e 1. \\ \hline
{\tt norm(A,a)} & Calcula uma norma de v. a pode ser 1, 2, `inf' ou \\
& `fro'. Pode-se omitir a, obtendo assim a norma 2. \\ \hline
{\tt diag(A)} & Extrai a diagonal da matriz A, e p\~oe em um vetor. \\ \hline
{\tt diag(v)} & Cria uma matriz diagonal, com o vetor v na diagonal. \\ \hline
{\tt trace(A)} & Calcula o tra\c{c}o de A. \\ \hline
{\tt [m,n] = size(A)} & Retorna o tamanho de A. \\ \hline

\end{tabular}
\end{center}}

\pause

O Maltab tamb\'em trabalha com matrizes esparsas. Para transformar uma matriz em uma matriz esparsa utilize {\tt sparse(A)}. Para mais informa\c{c}\~oes digite {\tt help sparse}.

\end{frame}

\section[Exerc\'icios]{}

\begin{frame}[fragile]
\frametitle{Exerc\'icios}

\begin{enumerate}
\item Crie um vetor aleat\'orio de 10 posi\c{c}\~oes. Calcule a norma desse vetor e o produto interno desse vetor com o vetor $e = (1,1,\dots,1)^T$.
\item Crie um vetor aleat\'orio de 10 posi\c{c}\~oes. Calcule $$S = \sum_{i=1}^{10}\sin^2(v_i),$$ onde $v_i$ \'e o i-\'esimo elemento desse vetor.
\item Cria uma matriz aleat\'oria 5 por 8. Extraia a submatriz 3 por 5 entre as colunas 3 e 7 e as linhas 2 e 4. Inclua uma linha nessa matriz, com os elementos 4,2,0,-2,-4, nessa ordem. Extraia a submatriz referente \`as \'ultimas 3 linhas e colunas dessa matriz. Calcule o determinante dessa submatriz.
\end{enumerate}

\end{frame}

\end{document}